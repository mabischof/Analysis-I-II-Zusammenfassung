% !TeX root = ../../ZF_bmicha_Ana.tex
\subsection{Eigenschaften \texorpdfstring{\hfill S.54}{S.54}}
    Eine Funktion $f : A \to B$ ist eine Vorschrift, die jedem $x \in A$ ein Element $f(x) \in B$ zuordnet, $f: x \to f(x)$.
    \begin{description}
        \item[Definitionsbereich:] $D(f) = A$
        \item[Zielbereich:] $Z(f) = B$
        \item[Wertebereich:] $W(f) = \{ f(x) \vert \ x \in D(f)\}$   
    \end{description}
    
    \textit{Hinweis: Lücken im Defintionsbereich müssen bei der Diskussion der Stetigkeit nicht betrachtet werden.\\ Bsp:}
    \small{$$ f(x) = \frac{1}{x} \hspace{8pt} \textrm{ist in} \hspace{8pt} \mathbb{R} \setminus \{-1, 1 \} \hspace{8pt} \textrm{stetig}$$}
    
    \subsubsubsection{Surjektiv}\vspace{3pt}
    
        Jeder Wert im Zielbereich $Z(f)$ wird angenommen.
        
        \mathbox{
            W(f) = Z(f)
        }
    \begin{samepage}
         \subsubsubsection{Injektiv}\vspace{3pt}
        Jede Horizontale schneidet den Graphen $\Gamma(f)$ höchstens einmal.
        \begin{itemize}
            \item $f(x_1) = f(x_2) \Rightarrow x_1 = x_2$, sonst nicht injektiv
        \end{itemize}
    \end{samepage}
        
    \subsubsubsection{Bijektiv}\vspace{3pt}
    
        \begin{center}
            Injektiv \& Surjektiv $\Leftrightarrow$ Bijektiv $\Leftrightarrow$ Umkehrbar
        \end{center}
    \subsubsubsection{Inverse Funktion}\vspace{3pt}
    
        Sei $f(x)$ eine Funktion von $D(f)$ nach $W(f)$, dann ist $f^{-1}: W(f) \to D(f)$ mit $y \mapsto f^{-1}(y)$ die inverse Funktion von $f(x)$.
        \begin{itemize}
            \item $W(f^{-1}) = D(f)$
            \item $D(f^{-1}) = W(f)$
        \end{itemize}
    \subsubsubsection{Gerade \& Ungerade}%\vspace{3pt}
        \begin{description}
            \item[gerade:]\phantom{as} $f(-x) = f(x)$ 
            \item[ungerade:] $f(-x) = -f(x)$ 
        \end{description}
    \subsubsubsection{Stetigkeit}\vspace{3pt}
    
        $f(x)$ ist stetig im Punkt $\xi$ falls
        $$
            \lim_{x\to\xi^-} f(x) = f(\xi) = \lim_{x\to\xi^+} f(x).
        $$
        \begin{itemize}
            \item Bei Lücken in $D(f)$ werden die einzelnen Abschnitte separat betrachtet.
        \end{itemize}
        
    \subsubsubsection{Monotonie}\vspace{3pt}
    
        \textbf{(Strikt) Monoton Steigend}
            \begin{itemize}
                \item $x_1 < x_2\ \Longleftrightarrow\ f(x_1) \leq f(x_2)$ \hfill (strikt: $<$)
                \item $f'(x) \geq 0$ \hfill (strikt: $>$)
            \end{itemize}
        \textbf{(Strikt) Monoton Fallend}
            \begin{itemize}
                \item $x_1 < x_2\ \Longleftrightarrow\ f(x_1) \geq f(x_2)$ \hfill (strikt: $>$)
                \item $f'(x) \leq 0$ \hfill (strikt: $<$)
            \end{itemize}
    \subsubsubsection{Beschränktheit}\vspace{3pt}
    
        Alle Funktionswerte sind in einem endlich breiten waagerechten Parallelstreifen enthalten.

