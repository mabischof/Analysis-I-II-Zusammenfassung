% !TeX root = ../../ZF_bmicha_Ana.tex
\subsection{Krümmung \texorpdfstring{\hfill S.66-67}{S.66-67}}
\vspace{3pt}

    \begin{itemize}
        \item Parametrisierung $\rvec(t) = (x(t),y(t))^T$
            $$
                k(t) = \frac{\dxt \ddyt - \ddxt \dyt}{\left( \dxt^2 + \dyt^2 \right)^{\frac{3}{2}}}
            $$
        \item Daraus folgt für die Parametrisierung $y = f(x)$:
        $$
            k(x)= \frac{f''(x)}{(1+f'(x)^2)^{3/2}}
        $$
        \item Polar-Parametrisierung $ \rho = f(\phi)$
            $$
                k(\phi) = \frac{(f(\phi))^2 + 2(f'(\phi))^2 - f(\phi) f''(\phi)}{[(f(\phi))^2 + (f'(\phi))^2]^{3/2}}
            $$
        
    \end{itemize}

    \vspace{5pt}
    
    \textit{$|k(t)|$ wird kleiner für abnehmende Krümmung d.h. grösserer Abstand zwischen Kurve und 
    Evolute.}