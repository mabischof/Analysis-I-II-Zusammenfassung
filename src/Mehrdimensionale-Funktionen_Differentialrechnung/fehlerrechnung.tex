% !TeX root = ../../ZF_bmicha_Ana.tex
\subsection{Fehlerrechnung \texorpdfstring{\hfill S.64}{S.64}}
    Die berechnete Grösse $f$ ist abhängig von den gemessenen Grössen $x,y$.
    Die gemessenen Grössen weichen mit den Messfehlern $dx,dy$ (auch $ \varDelta  x, \varDelta  y$ o.ä.) von der Realität ab.
    \begin{itemize}
        \item \textbf{Totales Differential / Absoluter Fehler}
            \mathbox{
                df \approx f_x\ dx + f_y\ dy
            }
        \item \textbf{Relativer Fehler}
            \mathbox{
                \frac{df}{f}
            }
    \end{itemize}
    \subsubsubsection{Bemerkungen}
        \vspace{0.5em}
        \begin{minipage}{0.54\linewidth}
            \centering \vspace{4pt}
            $1\%$ Genauigkeit
            $$
                \frac{dx}{x} = 1\% = \frac{1}{100}
            $$          
        \end{minipage}
        \begin{minipage}{0.45\linewidth}
            \centering
            Messfehler von $1^\circ$
            $$
                d\alpha = \frac{\pi}{180}
            $$
        \end{minipage}
        \vspace{3mm}\\
        \textit{Hinweis: in alle Formeln die Messwerte einsetzen, auch beim relativen Fehler. Auch wenn sie fehlerbehaftet sind.}