% !TeX root = ../../ZF_bmicha_Ana.tex
\subsection{Extremalstellen von \texorpdfstring{$f(x,y)$}{f(x,y)}}
    \vspace{0.25em}
    \begin{enumerate}
        \item Inneres untersuchen $\to$ $\textrm{grad}f \overset{!}{=} 0$
        \item Rand untersuchen
        \begin{itemize}
            \item Lagrange Multiplikatoren
            \begin{enumerate}
                \item $g(x,y)$ beschreibt Rand
                \item $\textrm{grad}f(x_o,y_o) = \lambda \cdot \textrm{grad}\, g(x_o,y_o)$\\[0.25em]
                      \phantom{llll}$\textrm{grad}\, g(x_o,y_o) \neq 0,\phantom{ll} \lambda \in \mathbb{R}$
                \item Gleichungssystem aus (a) und (b) lösen.
            \end{enumerate}
            \item Parametrisierung
            \begin{enumerate}
                \item Rand parametrisieren
                \item Parametrisierung in $f$ einsetzen
                \item Nach Parameter ableiten und nullsetzen.\\[0.25em] \phantom{llll}$f'(t) = 0$
            \end{enumerate}
        \end{itemize}
        \item Eckpunkte untersuchen
        \item Kandidaten vergleichen
    \end{enumerate}
    
\subsubsection{Verhalten an einer Extremalstelle} 

\begin{enumerate}
    \item Hesse-Matrix bilden: 
    $$ 
     H_f(x,y) = \begin{pmatrix} f_{xx} & f_{xy} \\ f_{yx} & f_{yy} \end{pmatrix}
    $$
    \item Die Koordinaten der ermittelten Extremalstelle einsetzen. Man erhält eine quadratische, symmetrische Matrix, da $f_{xy} \overset{!}{=} f_{xy}$. 
    \item Eigenwerte berechen: 
    $$
    \lambda_{1,2} = m \pm \sqrt{m^2 - p} 
    $$
    $ m = \frac{h_{11} + h_{22}}{2} \hspace{15pt} p = det(H) $
    \item Definitheit prüfen: 
    \begin{itemize}
        \item $\lambda_1, \lambda_2 > 0 \Rightarrow$ lokales Minimum
        \item $\lambda_1, \lambda_2 < 0 \Rightarrow $ lokales Maximum
        \item indefinit $\Rightarrow$ Sattelpunkt
        \item semidefinit $\Rightarrow$ keine Voraussage möglich
    \end{itemize}
\end{enumerate}